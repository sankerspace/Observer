\chapter{Hardware Implementation}

% **************************** Define Graphics Path **************************
\ifpdf
    \graphicspath{{Chapter3/Figs/Raster/}{Chapter3/Figs/PDF/}{Chapter3/Figs/}}
\else
    \graphicspath{{Chapter3/Figs/Vector/}{Chapter3/Figs/}}
\fi

The following sections shows us an overview about the hardware implementation of the Invariant Observer Design.
At first we get a description about the hardware platform on which the observer runs and some details
about the synthesis tools.
At the end of this chapter, there comes a design view about the synthesized hardware and a discussion about 
the design steps.
\section{Hardware Platform for Prototype}
The Invariant Observer is synthesized on a Field Programmable Gate Array(FPGA),on this platform it was possible to get an insight
about the runtime behaviour and to see the observers in action. 
The FPGA Board that was used for the simulations is a \textbf{DE2-115} Board from ALTERA\cite{altera1}.
This FPGA Board is ideal to illustrate fundamental concepts and advanced designs,and it gives us a possibiliy
to meet the necessary real-time requirements we need. The DE2-115 is equipped with a \textbf{Cyclone IV EP4CE115F29} 
FPGA Chip and it is powerful enough to emulate a CPU(Nios).In \cite{RTFMBJ13} this Fpga board was used for performance studies,besides other FPGA's,
so it was logical to use a similar environment.
The Board has an onboard oscillator of 50 Mhz,and with the use of a Phase Locked Loops(PLL) it is possible to increase the frequency for tests.
In the following section we will see how the increase of the frequency changes the design on Register Transfer level(RTL) and why bad design decisions 
can influence the maximal speed of the design.


\section{Synthesis and Design  }
This section gives us an overview about the synthesis tool and details about the Register Transfer Level design view.
An interesting part in this section is to see how the synthesis tool creates hardware structures based on the code of the
Observer VHDL implementation.\newline
For synthesis and compilation of the observer vhdl code ,the \textbf{Altera Quartus II Version 12.1 Build 177 Full Version} was used.
\subsection{Design View of the Register Transfer Level}
