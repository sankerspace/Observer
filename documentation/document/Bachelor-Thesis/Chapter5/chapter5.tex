\chapter{Summary}
\label{chapter:5}
%% NO PERSONAL FORMS us,we,I
%% quotes use like : ``text''
%% after point,doublepoints and commas there must be a BLANK SPACE

% **************************** Define Graphics Path **************************
\ifpdf
    \graphicspath{{Chapter3/Figs/Raster/}{Chapter3/Figs/PDF/}{Chapter3/Figs/}}
\else
    \graphicspath{{Chapter3/Figs/Vector/}{Chapter3/Figs/}}
\fi

It was introduced an alternative version of an Invariant Observer from \cite{RTFMBJ13} which circumvents specific design restrictions from \cite{RTFMBJ13}.
For example, propositions $\phi$ are not restricted by the logahedron class of atomic proposition like in \cite{RTFMBJ13}, so there is not a tight time bound to fulfil. 
With the possibility of a massively parallel constuction of the Observer Stages as parts of a complete Invariant Observer which allows evaluating a proposition $\phi$ 
at every clock cycle, it could result to a big performance step.
The Observer Stages are in way more flexible than in \cite{RTFMBJ13}, besides the possibility to change the Invariant $\tau$ during runtime. 
But these agile advantages should be tested in a more detailed experiment and under real-time conditions.
Another point to mention is, that the experiments considered only the case that the Observer Stages are running in the same time domain like the system that evaluates the proposition $\phi$ to W($\phi$).
For example three clock cycles which passes for W($\phi$) are also the same three clock cycles for the Observer Stages. It came out, that a better resolution could be performed if
the frequency that drives the Observer Stages is multiple higher than the frequency of the system that evaluates W($\phi$).
For example if the evaluation system for W($\phi$) is driven with 10Khz and the Observer Stages with 20Khz and an Invariance of $\tau = 2$ on W($\phi$) should be monitored, then the Observer Stages must observe an invariance
of $\tau = 4$ to meet the same time domain. That aspect should be considered in further real-time experiments. 
In \cite{RTFMBJ13} these two systems are acting together.
The Invariant Observer should be considered as a temporal operator which follows the specifications of ptMTL $\boxdot _{[ y,y+\tau ]}$ but with the restriction that it can not perform the observation of the 
invariance inside intervall [0,y].
That means, that the computation time for a  proposition $\phi$ can not be less than y (maybe the only disadvantage). 
The design approach of the Invariant Observer shows maybe possibilities to implement other temporal operators from the Metric Temporal Logic like the ``Exists-Operator within Intervals'', 
that could be also an attempt for further studies.




