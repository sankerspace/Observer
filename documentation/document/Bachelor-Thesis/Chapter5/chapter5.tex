\chapter{Summary}
\label{chapter:5}
%% NO PERSONAL FORMS us,we,I
%% quotes use like : ``text''
%% after point,doublepoints and commas there must be a BLANK SPACE

% **************************** Define Graphics Path **************************
\ifpdf
    \graphicspath{{Chapter3/Figs/Raster/}{Chapter3/Figs/PDF/}{Chapter3/Figs/}}
\else
    \graphicspath{{Chapter3/Figs/Vector/}{Chapter3/Figs/}}
\fi

It was introduced a alternative version of an Invariant Observer from \cite{RTFMBJ13} which circumvents specifical design restrictions from \cite{RTFMBJ13}.
For example no restrictions on the evaluation time y for a proposition $\phi$. 
With the possiblity of a massively parallel constuction of the Observer Stages as part of a complete Invariant Observer which allows to evaluate a proposition $\phi$ 
at every clock cycle, it could result to a big performance step.
The Observer Stages are in way more flexible than in \cite{RTFMBJ13}, besides the possibility to change the Invariant $\tau$ during runtime. But these advantages should be
tested in a more detailed experiment and under real-time conditions.
Another point to mention is, that the experiments considered only the case that the Observer Stages are running in the same time domain like the system that evaluates the proposition $\phi$ to W($\phi$).
For example three clock cycles which passes for W($\phi$) are also the same three clock cycles for the Observer Stages. It came out, that a better resolution could be performed if
the frequency that drives the Observer Stages is multiple higher than the frequency of the system that evaluates W($\phi$).
For example if W($\phi$) is driven with 10Khz and the Observer Stages with 20Khz and a Invariance of $\tau = 2$ on W($\phi$) should be monitored, then the Observer Stages must observe a invariance
of $\tau = 4$ to meet the same time domain. That aspect should be considered in further real-time experiments. 
In \cite{RTFMBJ13} these two systems are acting together.
The Invariant Observer should be considered as a temporal operator which follows the specifications of ptMTL $\boxdot _{[ y,y+\tau ]}$ but with the restriction that it can not perform the observation of the 
invariance inside intervall [0,y], because the computation of proposition $\phi$ can not be immediately available.(remember y as the calculation time to calculate proposition $\phi$ to result W($\phi$ ))
Finally, the design approach of the Invariant Observer shows another possiblity to implement the other temporal operators from the Metric Temporal Logic like the ``Exists-Operator within Intervalls''.




