%*****************************************************************************************
%*********************************** Second Chapter **************************************
%*****************************************************************************************

\chapter{Software Implementation and Algorithm}

%\ifpdf
%    \graphicspath{{Chapter2/Figs/Raster/}{Chapter2/Figs/PDF/}{Chapter2/Figs/}}
%\else
%    \graphicspath{{Chapter2/Figs/Vector/}{Chapter2/Figs/}}
%\fi


\section{Algorithm of the Invariant Observer Stages}

The following algorithm shows the proper behaviour of an observer stage.\newline

\begin{algorithm}
\caption{Pseudo Code of an Observer Stage}
\label{alg:observerstage}
\begin{algorithmic}[1]
\REQUIRE Precondition: $m \ge y$ and clock = 0
\STATE Initialize: count = 0
\IF{(clock mod m) = 0} 
 \IF{W($\phi$) = 0} 
 \STATE  /*evaluates finished calculation of $\phi$ after m clock cycles*/
 \STATE  count = 0 
 \ELSE
  \STATE /*do nothing*/ 
 \ENDIF
\ENDIF 
\STATE /*Following code executes every clock cycle*/
\IF{count = $\tau+1$}
 \STATE output = 1
\ELSE
 \STATE output = 0
\ENDIF
\STATE count = min(count + 1,$\tau + 1$)
\RETURN output
\end{algorithmic}
\end{algorithm}

As you can see in  Algorithm~\ref{alg:observerstage} the algorithm is separated in two main parts. 
The upper part checks from the start of the observer stage, and periodically every m clock cycles,
the status of the current signal value W($\phi$). If W($\phi$) has an active status, there is no change of
the counter, but otherwise the counter will be reseted.
This is important that one observer stage recognize that the invariance qualification was not satisfied
at that time. If the conjunction of all observer stages is done, and at least on stage hast not an active output,
than the result is false, which means no invariance qualification fulfilled at the time of the binary add operation. \newline

The down part is executed at every clock cycle and counts up the counter to the maximum range of the invariance qualification.
If the counter reaches that maximum, his output is set to an active state. The counter remains in the maximum level $\tau+1$ if
nothing changes his value in the upper part of the algorithm. In fact, the counter represents the invariance qualification of length
$\tau$, plus 1 indicates that the present value must be involved in the invariance qualification.\newline

It should be mentioned that this current design does not implement or handle the calculations of the propositions $\phi$, 
which is indicated with W($\phi$). On the other hand the observers from \cite{RTFMBJ13} are responsible to take the necessary inputs,
calculate the atomic propositions (with ATCheckers) and evaluate immediately the ptMTL operator qualifications.  
These steps have to be done in a tight time bound.

  





